\chapter*{\centering Abstract}\label{secAbstrct}

Trading off risks against returns appears to be essential and vital for making a financial decision. Hence the econometric analysis of risk (volatility) becomes an important part in forecasting market tendency and supporting making financial decisions, such as portfolio diversification, risk management and derivative pricing. In the last 20 years volatility was a research hotspot in financial industry. Volatility is regarded as a parameter for evaluating the risk of assets return. Generally, the stronger the volatility is, the higher the risk is.

In the traditional financial models, the variance of the time series is always assumed as constant. However, it is found that the volatilities of financial time series have always the features of ``clustering'' and ``fat tails''  \citep{Mandelbrot1963,EugeneF.Fama1965}. These features obviously are not consistent with the assumption of constant, so the traditional econometric methods cannot analyze the financial time series efficiently in practice. To overcome this problem, several economists have carried out studies on researching and developing frameworks for evaluating volatility. Since Engle introduced the autoregressive conditional heteroskedasticity (ARCH) model \citep{Engle1982}, the extensions of ARCH model appeared and spread rapidly. Among the carried out researches, the Generalized ARCH (GARCH) model and its derivatives are most widely used \citep{Bollerslev1986}.

\gls{ny}, \gls{la} and \gls{un} are abbreviations whereas

