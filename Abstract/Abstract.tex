\chapter*{\centering Abstract}\label{secAbstrct}

The use of the GARCH model is widely observed in the empirical literatures. However, this model may cause the misspecification and assumes that the unconditional variance of the time series is constant. The recently proposed semi-parametric GARCH model, which composes of the conditional heteroskedasticity and the scale function, can improve the GARCH model. In this paper the definitions, the features and the estimation of the GARCH model, the Semi-GARCH model and their extensions are investigated. Based on the Semi-GARCH model the Semi-EGARCH and the Semi-CGARCH models are introduced originally in this work. In the empirical example the Semi-APARCH, the Semi-EGARCH and the Semi-CGARCH models are applied to the returns of Allianz and BMW at fixed trading time points. It is found that the semi-parametric models have more correct theoretical basis. They can model the conditional heteroskedasticity and the scale change at the same time. Furthermore, the semi-parametric models work well with the returns at fixed trading time points.

Key Words: GARCH, Semi-GARCH, Semi-APARCH, Semi-EGARCH, Semi-CGARCH, frequency returns, fixed trading time points.
