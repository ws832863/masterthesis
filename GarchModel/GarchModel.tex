\chapter{Parametric volatility models}\label{secGarchmodel}
It is known that, volatility clustering exists in financial time series, and the distribution of random variables appears the fat tails. Mandelbrot observed the pattern of thick tails or fat tails of stock returns fluctuation, which was not consistent with the traditional assumption of normal distribution, by study the stock return index \citep{Mandelbrot1963}. 
Fama suggested that the financial market returns had the property of volatility clustering. It means that volatility is not only dynamic but also clustered, which appears strong in one certain period but weak in another period. The reasons for these phenomena could be traced back to speculation, political changes, government monetary and fiscal policy, and etc. \citep{EugeneF.Fama1965}. ``Volatility clustering'' and ``fat tails'' cannot be explained by the traditional economic models, which assume the variance is constant, until the Autoregressive Conditional Heteroscedasticity (ARCH) model was introduced by Engle in 1982 \citep{Bollerslev1992}.


Different from the traditional models, ARCH model suggests that the conditional variance could change over time as a function of past errors with the unconditional variance remaining constant \citep{Engle1982}. ARCH model soon became an important tool of volatility measurement, because it improved the traditional model and fit reality better. However, there are some drawbacks in ARCH model. In practical applications of the ARCH model, a relatively long lag in the conditional variance equation is often required, which might lead to increase in complexity of estimating parameters and decrease the freedom degree. When the lag order is high, estimating a totally free lag distribution will often violate the constraint of non-negativity parameters. But the restrict condition is exactly needed in this model to ensure conditional variance to be non-negative \citep{Bollerslev1986}. Therefore, there are many economists tried to improve ARCH models. Among these researches, the generalized ARCH (GARCH) model, which is introduced by Bollerslev (1986), is the most widely well known one with a better framework to study time-varying volatility in financial markets. 

\section{The GARCH model}
GARCH model added the lagged conditional variances to ARCH model, so that it has a longer memory and a more flexible lag structure than ARCH model.

\subsection{Definitions and basic properties}
In GARCH model, $F_{t-1}$ denote the set of the past information. The GARCH process is presented in \ref{equ2.1}:

\begin{equation} 
\label{equ2.1}
	\varepsilon_t =  \eta_t h_t^{1/2},    \qquad    \varepsilon_{t}|F_{t-1}\sim N(0,h_{t})_,
 \end{equation} 
 
in which $\varepsilon_t$ is a random variables; $\eta_t$ are i.i.d. random  variables with zero mean and unit variance; $h_{t}$ means the conditional variance of $\varepsilon_{t}$ and is defined as follow:

\begin{equation}
\label{equ2.2} 
h_t = \omega + \sum_{i=0}^p \alpha_i \varepsilon_{t-1}^2 + \sum_{j=1}^q\beta_j h_{t-j},
\end{equation} 

where $\omega>0$ is the variances intercept parameter; $p>0$ is the order of ARCH parameters; $q \geq 0$ is the order of GARCH parameters; $\alpha$ and $\beta$ ( $ \alpha_i \geq 0, i=1, \dots ,p; \beta_j \geq 0, j=1,\dots,q$) are two vectors of unknown parameters respectively \citep{Bollerslev1986}.

From the equation \ref{equ2.2} we can see that, GARCH model contains the lagged conditional variances in addition to ARCH model. When  $q=0$, the process is ARCH model; if $ p = q = 0$, this process is a simple white noise; while $\sum_{i=1}^p\alpha_i <1$, this process is an infinite dimensional ARCH  $(\infty)$. This means that ARCH is a special form of GARCH process. Most of the cases, a high order ARCH model can be replaced by a low order GARCH model without breaking the constraint of non-negativity by estimating too many parameters. For example, In the application of ARCH (p) model, sometimes the value of p is very high; but for GARCH (p, q) model, $p=1, q=1$ are enough to achieve the same outcome \citep{Engle1986}.


In the paper from Engle, the stationary of GARCH model is introduced. As it is known, a process $\lbrace X_{t} \rbrace$  is weakly stationary,  if $i) E(X_{t}^{2}) < \infty$  for all  $t \in T$ ,$ ii) E(X_{t})$  and $cov(X_t,X_{t+s})$  are independent of $t$ for $s \in Z, t \in T$ \citep{Bougerol1992}. So the GARCH $(p,q)$ process with $E\left( \varepsilon_t \right) =0, Var(\varepsilon_t) = \omega(1-  \sum_{i=1}^p\alpha_i - \sum_{j=1}^q\beta_j )^{-1}$ and $cov(Y_t,Y_s)=0$ for $t \neq s$ is wide-sense stationary if and only if $\sum_{i=1}^p\alpha_i + \sum_{j=1}^q\beta_j < 1 $ \citep{Bollerslev1986}.

\subsection{Estimation of the GARCH model}

There are many methods to estimate the GARCH model, such as the Maximum Likelihood Estimation (MLE), the Quaxi-Maximum Likelihood Estimation (QMLE) and the Least Absolute Deviation Estimation (LADE), etc. Usually, the maximum likelihood estimation, which is introduced in this work, is most widely used. Besides the condition of stationary, to estimate the GARCH model a very important assumption is needed that the moment fourth order of this model exists. The conditions under which this moment exists are usually very complex and will not be discussed in this thesis. In the special cases, when the model is a normal GARCH (1,1) model,
 
\begin{equation}
E(\varepsilon_t^4)=\frac{3\omega^2(1+\alpha_1+\beta_1)}{(1-\alpha_1-\beta_1)(1-\beta_1^2-2\alpha_1\beta_1-3\alpha_1^2)}_.
\end{equation}

This condition is $3\alpha_1^2+2\alpha_1\beta_1+\beta_1^2<1$ \citep{Milhj2012}.  

Letting  $\theta=(\omega, \alpha_1, ..., \alpha_p,\beta_1,..., \beta_q)$, the conditional Gaussian log-likelihood function is defined as follows:

\begin{equation}
 L(\theta) = \frac{1}{n}\sum_{t-1}^nl_t,
\end{equation}

where

\begin{equation}
l_t =-\frac{1}{2}ln(h_t^2(F_{t-1};\theta))-\frac{\varepsilon_t^2}{2h_t^2(F_{t-1};\theta)}_.
\end{equation}

Then maximize $L(\theta)$ by calculating and putting the partial derivatives to zero. After n times iteration, the approximated value (MLE) of $\theta$ is obtained, which is denoted as $\hat{\theta}$.

The maximum likelihood method is used by most of cases in GARCH family. Based on the assumption of normal distribution, GARCH model can partially explain the clustering and fat tails properties of the returns. However, after the standardized by GARCH model, the residuals also have the properties of clustering and fat tails, which is not consistent with the normal assumption. So Bollerslev and Nelson used student-t distribution and generalized error distribution (GED) instead of normal distribution to estimate GARCH models \citep{Bollerslev1986,Nelson1991}. When we estimate other GARCH models, which will be discussed below as examples, the main method is the same, maximum likelihood method. The most significant difference is the assumption of the distribution. 

Although a simple GARCH model can solve the problem of the ARCH model with high-order linear declining lag structure, it also has some obvious drawbacks.  The two constraint conditions of GARCH model, parametric non-negativity and bounded, restrict its applicability. The assumption of non-negative parameters raises the difficulty of estimation. Also with this assumption, the GARCH model cannot well explain the asymmetry of volatility and leverage effects. In order to extend the application of GARCH model, many GARCH derivatives appeared aiming at different problems. For example, to explain the asymmetry of volatility, the exponential GARCH model (EGARCH) was proposed by Nelson \citep{Nelson1991}, and threshold GARCH (TGARCH) was suggested by Zakorian \citep{Zakoian1994}. Ding, Granger and Engle introduced the APARCH model \citep{Ding1993}, which are able to explain the leverage effects in financial market better. Engle and Lee defined the component GARCH (CGARCH) model to exactly interpret the shock influences to the long-run and the short-run volatility respectively \citep{0-19-829683-5}.

\section{The asymmetric power ARCH model}

Generally, if a data series follows the normal distribution, we can characterize it by its first two moments. However, if the error of the data follows a non-normal distribution, then higher moments of skewness, kurtosis have to be applied, which describes the data beyond to adequately. In this instance, other power transformations appeared to be more appropriate than squared term \citep{McKenzie1999}. Ding, Granger and Engle (1993) further studied the autocorrelation of square returns and absolute returns based on GARCH model and Taylor model. It is found that, long lags between absolute returns are more correlated than the returns themselves for the return process \citep{stephen1986modelling}. Furthermore, $|r_{t}|^{d}$ has the largest autocorrelation when d is around 1, but smaller monotonically one when d deviates from 1. Based on this they imposed an asymmetric GARCH model, asymmetric power ARCH model, which enables the power of the heteroskedasticity equation to be estimated from the data.


The asymmetric power ARCH model is defined by

\begin{equation}
h_{t}^{\delta/2} = \omega + \sum_{i=1}^{p} \alpha_{i}(|\varepsilon_{t-i}|-\gamma_{i}\varepsilon_{t-i})^{\delta} + \sum_{j=1}^{q}\beta_{j}h_{t-j}^{\delta/2},
\end{equation}

where $\omega>0, \delta\geq0, \alpha_{i}\geq0, i=1, \ldots, p, -1<\gamma_{i}<1, i=1, \ldots, p, \beta_{j}\geq0.j=1, \ldots, q$. $\gamma$ is the leverage parameter, which takes the asymmetric news impact on the volatility into account,  and $\delta$ is the parameter for the power term, which is a suitable positive number. When $\gamma$ is quite smaller than 1, the leverage effect is not strong. While the value of $\gamma$  is closer to 1, the leverage effect is stronger. This means that the contribution of a negative return on yesterday to today's volatility is more than the contribution of a positive return. When $\gamma$ is equal to 1, there is a perfect leverage effect, which means that a positive return on yesterday does not affect today's volatility at all \citep{FengYuanhua;Sun2013}.

This model exhibits a power transformation of the conditional standard deviation process, which can linearize nonlinear models. It also reveals the asymmetric absolute residuals, which can reflect the leverage effect of the stock market returns.


In APARCH model, when $\gamma_{t}$ is conditional normal, and \[\frac{1}{\sqrt{2\pi}}\sum_{i=1}^{p}\alpha_{i} [\left(1 + \gamma_{i} \right)^\delta + (1-\gamma)^{\delta}]2^{\frac{\delta-1}{2}}\Gamma (\frac{\delta+1}{2}) + \sum_{j=1}^{q}\beta_{j}<1,\]
which is the condition to ensure the existence of $Eh_{t}^{\delta/2}$ and $E|\varepsilon_{t}|^{\delta}$, then when $\delta\geq2$, $\varepsilon_t$   is covariance stationary (sufficient condition) \citep{Ding1993}.


This model includes several ARCH models as a special case. If the values of $\delta$ and $\gamma_{i}$ are changed, APARCH model derives into the following models, the standard GARCH model, the GJR-GARCH model, the TS-GARCH model(Taylor and Schwert model), the NGARCH model (Nonlinear GARCH model) and the TGARCH model (threshold GARCH model).

When $\delta=2,\gamma_{i}=0$, the APARCH model turns into a GARCH model with the covariance stationary condition for  $\varepsilon_{t}$ as $\sum_{i=1}^{p}\alpha_{i} + \sum_{j=1}^{q}\beta_{j}<1$ \citep{Bollerslev1986}.

When $\delta = 2, \gamma_{i}\neq 0$, the APARCH model can be named as the GJR (Glosten, Jagannathan and Runkle, 1993) model 
\[h_{t}=\omega + \sum_{i=1}^{p}\alpha_{i}\varepsilon_{t-i}^{2}+\gamma_{i}\varepsilon_{t-i}^{2}I(\varepsilon_{t-i}<0)+\sum_{j=1}^{q}\beta
_{j}h_{t-j}\]

with the  covariance stationary condition for $\varepsilon_{t}$ as $\sum_{i=1}^{p}\alpha_{i}[1+\gamma_{i}^{2}]+\sum_{j=1}^{q}\beta_{j}<1$.

This model uses the indicator function I to simulate the asymmetric influence of the positive and negative shocks on the conditional variance \citep{Glosten1993}.

When $\delta=1,\gamma_{i}=0$, the APARCH model transforms into the TS-GARCH model
\[h_{t}^{1/2} = \omega + \sum_{i=1}^{p}\alpha_{i}|\varepsilon_{t-i}|+\sum_{j=1}^{q}\beta_{j}h_{t-j}^{1/2}\]

with the covariance stationary condition for  $\varepsilon_{t}$ as  $\sqrt{\frac{2}{\pi}} \sum_{i=1}^{p}\alpha_{i} + \sum_{j=1}^{\alpha}\beta_{j}<1$
\citep{Schwert1990,stephen1986modelling}.

When $\delta=1,\gamma_{i}\neq0$, a asymmetric Taylor/Schwert model is obtained, which is named as the TGARCH (threshold GARCH) model

\[h_{t}^{1/2} = \omega + \sum_{i=1}^{p}\alpha_{i}|\varepsilon_{t-i}|+\gamma_i|\varepsilon_{t-i}|I(\varepsilon_{t-i}<0)+\sum_{j=1}^{q}\beta_{j}h_{t-j}^{1/2}\]

 with the covariance stationary condition for  $\varepsilon_{t}$, the same TS-GARCH model,as  $\sqrt{\frac{2}{\pi}} \sum_{i=1}^{p}\alpha_{i} + \sum_{j=1}^{\alpha}\beta_{j}<1$\citep{Zakoian1994}.


\section{The exponential GARCH model}

It is considered that, GARCH model has taken the focus on the magnitude matters of excess return, but ignored the information of directions. Moreover, it is proven by empirical evidence that the direction does impact on volatility. Especially for broad-based equity indices and bond market indices, it is observed that market declines forecast higher volatility than comparable market increases do \citep{Engle2001}. In addition, the estimated coefficients often violate the parameter restrictions in GARCH model. It is also found that the left tail of the returns distribution was fatter than the right one. The reason is generally considered as the asymmetric response of investors to “good news” and “bad news” \citep{Jondeau2003}. The existence of the asymmetric effect shows that there are some problems for the constraint condition for the  conditional variance. Nelson imposed the exponential GARCH model in 1991, which provides a good solution for the problems of GARCH. 

EGARCH model ensure that $h_{t}$  remains nonnegative by making $ln(h_{t})$ linear. It does not need any restriction on the parameters to guarantee the positivity of the variance.

The exponential GARCH model can be written as follows:

\begin{equation}
ln(h_t) =\omega + \sum_{i=1}^p(\alpha_i\eta_{t-i}+\gamma(|\eta_{t-i}|-E|\eta_{t-i}|))+\sum_{j=1}^q\beta_jln(h_{t-j}).
\end{equation}

In this model, $\gamma[|\eta_t|-E|\eta_t|]$ that define the size effect, and  $\alpha\eta_t$   that defines the sign effect of the shocks on volatility. The size effect is a typical ARCH effect, but the sign effect is asymmetrical. The value of $\alpha$ represents the existence of the asymmetry.  When $\alpha$ is not equal to 0, there is the asymmetric effect.  For example, the leverage effect, which has imposed by Black in 1976. When $\alpha<0$, the ``bad news'' and ``good news'' may cause the different influence of the volatility. The innovation in conditional variance is now positive (negative) when returns innovations are negative (positive) \citep{Nelson1991}.

According to the Theorem 2.1 from Nelson and the definition of the stationary, when $\gamma$  and $\theta$  do not equal to zero at the same time, the EGARCH process of order (1,1) is strictly stationary and ergodic if and only if $\alpha_1^2 < \infty$ and $|\beta_1|<1$. 

The estimation of EGARCH model is mostly same as the estimation of GARCH model. The difference between them is that, in EGARCH model, the generalized error distribution (GED) is used. Because this distribution includes the normal as special case as well as many other distributions, some of which are fatter tailed than the normal one and some are thinner tailed; and when the process is strictly stationary and the distribution of  $\eta_t$ is not too fat tailed, $h_t$ and $\varepsilon_t$ have finite unconditional moments of arbitrary order \citep{Nelson1991}.

\section{The component GARCH model}

The traditional GARCH model has not distinguished the long-term and the shore-term component. However, it is known that the stock prices always fluctuate around an average value. This phenomenon is called mean-revert. It is also found that mean-revert of short-term volatility is more rapid than for the long-term one and the market volatility must have enough persistence to influence the stock returns in the long-run \citep{XinzhongXuandStephenJ.Taylor1994}. In order to explain this different response between short-term and long-term, Engle and Lee (1999) imposed component GARCH model. In this model, the conditional variance is decomposed into permanent and transitory component. So that the model can be used to investigate the long-run and short-run movements of volatility affecting securities.

The component GARCH model is defined by

\begin{equation}
h_{t}=q_{t}+\sum_{i=1}^{p}\alpha_{i}(\varepsilon_{t-i}^{2}-q_{t-i}) + \sum_{j=1}^{q}\beta_{j}(h_{t-j}-q_{t-j}),
\end{equation}

and
\begin{equation}
q_{t} = \omega + \rho q_{t-1}  +\varphi(\varepsilon_{t-1}^{2}-h_{t-1}),
\end{equation}

where $q_{t}$ the permanent component of the conditional variance and $(h_{t-j}-q_{t-j})$ is the transitory component of the conditional variance \citep{0-19-829683-5,Ghalanos2011}. 

For the CGARCH (1,1) model, $(\alpha_1 +  \beta_1)$ is the mean-revert of $s_{t}$. In terms of economy, the smaller the mean-reverting rate, the less persistent the expected volatility to market shocks in the past. That means, when the market receives the information of shocks, volatility responses quickly but has lower persistence. If $0<(\alpha_1 +  \beta_1)<1$, the mean-reverts of short-run volatility component is zero at a geometric rate of  $(\alpha_1 +  \beta_1)$;  If $0<(\alpha_1 +  \beta_1)\ll1$, the impact of volatility shocks on the short-run volatility component is short-lived. The parameters $\rho$  is used to examine the persistence of shock impacts on the long-run component. If $0<\rho<1$,  the long-run volatility component follows an AR process, and will converge to a constant level defined by $\omega/(1-\rho)$. When  $\rho$ is extremely close to 1, usually between 0.99 and 1, the long-run volatility component converge to $\omega$ very slowly; If $0<(\alpha_1 +  \beta_1)<\rho<1$, the impact of volatility shocks on the long-run volatility component diminish as well but be more persistent than that of the short-run component. $\varphi$ shows the sensitivity of the  long-run component to volatility shocks. $\alpha$ expresses shows the sensitivity of the  short-run component to volatility shocks. When $\alpha\geq\varphi$, the immediate impact of volatility shocks on the long-run component would be smaller than that on the short-run component.

In GARCH model, the conditional variance must be non-negative. This condition is also required in component GARCH model. But this is not demanded by $s_{t}$, which can be positive or negative over time to exhibit the self-correction feature of the mean-reverting volatility process. The condition of stationary in GARCH(1,1) model is  $\alpha_{1} + \beta_{1}<1$ , besides which the stationary condition of CGARCH(1,1) model also includes $(\alpha + \beta)(1-\rho)+\rho<1$ and requires $\rho<1,(\alpha + \beta)<1$ in component model. When $\rho=1$, the mean series is not a covariance stationary process, but also is still a strictly stationary and ergodic process. Therefore, the stationary conditions of component GARCH(1,1) model can be concluded as $0<(\alpha + \beta)< \rho<1, 0<\varphi<\beta,\alpha>0,\beta>0,\alpha_{0}>0,\varphi>0$, But they are just sufficient conditions, not necessary ones \citep{0-19-829683-5}.