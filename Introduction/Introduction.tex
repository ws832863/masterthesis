\chapter{Introduction}\label{secIntroduction}
Trading off risks against returns appears to be essential and vital for making a financial decision. Hence the econometric analysis of risk (volatility) becomes an important part in forecasting market tendency and supporting making financial decisions, such as portfolio diversification, risk management and derivative pricing. In the last 20 years volatility was a research hotspot in financial industry. Volatility is regarded as a parameter for evaluating the risk of assets return. Generally, the stronger the volatility is, the higher the risk is.

In the traditional financial models, the variance of the time series is always assumed as constant. However, it is found that the volatilities of financial time series have always the features of ``clustering'' and ``fat tails'' \citep{Mandelbrot1963,EugeneF.Fama1965}. These features obviously are not consistent with the assumption of constant, so the traditional econometric methods cannot analyze the financial time series efficiently in practice. To overcome this problem, several economists have carried out studies on researching and developing frameworks for evaluating volatility. Since Engle introduced the autoregressive conditional heteroskedasticity (ARCH) model \citep{Engle1982}, the extensions of ARCH model appeared and spread rapidly. Among the carried out researches, the Generalized ARCH (GARCH) model and its derivatives are most widely used \citep{Bollerslev1986}.

 According to many studies \citep{Gourieroux1992,Eubank1993}, in parametric model, the preselected model might be too restricted or too low-dimensional, which may not fit unexpected features and cause the misspecification. However, in nonparametric model, the parameters of the model cannot be estimated, and the model cannot be explained due to lack of specific functions. Instead of parametric model or nonparametric model, recently proposed semi-parametric model will be introduced in details in this paper, which introduces a smooth scale function into the standard GARCH model. This model does not need a prespecified function and is less sensitive to model misspecification. At the same time, the model can be also explained \citep{Di2011}.

In this paper, the definition, estimation, some properties of semi-parametric model and the methods of bandwidth selection are discussed. Furthermore, based on the study of the semi-parametric GARCH model and semi-parametric asymmetric power ARCH model, which are introduced by Feng \citep{Feng2004,FengYuanhua;Sun2013}, the semi-parametric exponential GARCH and component GARCH models are originally defined. Then, the discussed semi-parametric models, i.e. Semi-APARCH, Semi-EGARCH and Semi-CGARCH models, are applied to the returns of BMW and Allianz from January 2006 to September 2014. Different from other papers, to get the more exact analyzing results the fixed trading time points are used in this paper.

The scope of this paper is as follows. In section 2, the parametric models are introduced. The semi-parametric models are described in section 3. Section 4 reports the application of the semi-parametric models to the returns of BMW and Allianz and the empirical results on the volatility of the selected data sets. Finally, this paper is concluded in section 5.