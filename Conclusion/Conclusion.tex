\chapter{Conclusion}

In this paper several parametric and semi-parametric models are introduced; then the semi-parametric models are applied to the high-frequency returns at fixed trading time points. The semi-parametric model has better theory and works well in practice. The parametric model does not distinguish the scale change and conditional heteroskedasticity. It can simulate the effect of financial crisis directly. However, this model considers the nonstationary financial time series as the stationary time series, which may cause the inexact estimation and forecast. The semi-parametric model introduces a smooth scale function into the standard GARCH model. Therefore, the conditional heteroskedasticity and scale change in a financial time series can be modeled at the same time.

The application shows that the semi-parametric model works well with the fixed trading time points. It not only can express the trend of the returns, but also the different level of the volatility at the different trading time points. The semi-APARCH and semi-EGARCH model can show up the leverage effect and in the semi-APARCH model the leverage effect is stronger. The semi-CGARCH model shows that the immediate impact of shocks on the short-run component is larger; but the persistence in the long-run is larger.

There are still some questions in this paper. Because of the localization of R, the volatility of semi-CGARCH cannot be used to analyze the persistence of shocks on short-run and long-run. Moreover, because there is not enough time to collect more data, the level of the volatility at the different fixed trading time points cannot be discussed in detail. These questions will be discussed in the future.
